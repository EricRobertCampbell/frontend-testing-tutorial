\documentclass[t]{beamer}

\usetheme{metropolis}

\usepackage{url}
\usepackage{lstautogobble}
\usepackage{listings}
\lstset{numbers=left, frame=l, tabsize=3, basicstyle=\footnotesize }

\title{Frontend Testing with Jest and Cypress}
\author{Eric Campbell}
\date{}

\begin{document}
	\begin{frame}
		\maketitle
	\end{frame}

	\section{Testing Fundamentals}
	\begin{frame}{Testing Fundamentals}
		\begin{itemize}
			\item ``3'' Types
				\begin{enumerate}
					\item Unit Tests: Test one unit (e.g. a single function)
					\item Integration Tests: Tests how several different units integrate with each other (e.g.  single portion of the application)
					\item End-to-End Test: Test the entirety of an application from one end to another
				\end{enumerate}
			\item Basic approach: AAA
				\begin{itemize}
					\item \textbf{A}rrange: Set up the initial conditions of the test
					\item \textbf{A}ct: Do / change something
					\item \textbf{A}ssert: use assertions to guarantee that the correct thing happened (or not)
				\end{itemize}
			\item Test Runner:  a program that takes care of running our tests for us (e.g. when to run or re-run tests, how to display the results, how to find the tests, \&c.)
		\end{itemize}
	\end{frame}

	\begin{frame}{Testing Solutions}
		\begin{itemize}
			\item Unit and Small Integration Tests: Jest (\url{https://jestjs.io/})
			\item Larger Integration and End-to-End: Cypress (\url{https://www.cypress.io/})
			\item $+$ some other smaller libraries to enhance each of them
		\end{itemize}
	\end{frame}

	\section{Jest --- Unit Tests}
	\begin{frame}[fragile]{Jest}
		\begin{itemize}
			\item Jest: test runner, also provides some tests
			\item \texttt{describe}, \texttt{it/test}, \texttt{expect}
			\item Basic form:
				\begin{lstlisting}[autogobble]
					describe("The thing I'm testing", () => {
						test('should do a thing', () => {
							expect(something).toBe(the correct value)
						})
					})
				\end{lstlisting}
		\end{itemize}
	\end{frame}

	\begin{frame}[fragile]{Running Jest}
		\begin{itemize}
			\item Will automatically pick up \texttt{*.test.js} or \texttt{*.spec.js} files
			\item \texttt{npm run test} --- all of them
			\item \texttt{npm run test PATH\_TO\_FILE} --- will only run that one
			\item Watches files for changes and re-runs tests as appropriate
			\item By default, it will only re-run the failing ones
			\item Important assertions (all of these also take \texttt{.not.????} to negate them):
				\begin{itemize}
					\item \texttt{toBe}: test equality
					\item \texttt{toBeUndefined}: is it \texttt{===undefined}
					\item \texttt{toMatchSnapshot}: does it match the stored snapshot? (You can pass in basically anything; the first time a snapshot is generated)
					\item \texttt{toThrow}: does it throw the correct error?
				\end{itemize}
			\item (Full list: \url{https://jestjs.io/docs/en/expect})
		\end{itemize}
	\end{frame}

	\begin{frame}{Example Program}
		Our test program is a simple React application that uses inputs to ask the user for two numbers and an operation, and displays the result. Let's test the functions in the \texttt{functions.js} file.
	\end{frame}
	\begin{frame}[fragile, allowframebreaks]{Example --- Testing a Single Function with Jest}
		word
	\end{frame}

	\begin{frame}[fragile]{Testing DOM Elements with \texttt{react-testing-library}}
		\begin{itemize}
			\item Use \texttt{react-testing-library} (\url{https://testing-library.com/docs/react-testing-library/intro/})
			\item Philosophy: tests should resemble the way users interact with the site
			\item Uses the node virtual DOM to run applications
			\item Adds several commands to make assertions about the DOM 
			\item \texttt{npm install --save-dev @testing-library/react}
			\item At the top of your testing file:
				\begin{lstlisting}[autogobble]
					import { render, screen } from @testing-library/react 
					import @testing-library/jest-dom/extend-expect
				\end{lstlisting}
		\end{itemize}
	\end{frame}

	\begin{frame}[fragile]{Firing Events}
		\begin{itemize}
			\item Use \texttt{@teseting-library/user-event} for events
			\item (\texttt{react-testing-library} has its own way to fire events, but this library is `more realistic')
			\item \texttt{npm install --save-dev @testing-library/user-event}
			\item In your file: \texttt{import userEvent} from '@testing-library/user-event'
			\item Fairly intuitive interface:
				\begin{lstlisting}[autogobble]
					userEvent.click(the thing to click)
					userEvent.type(element, 'here{enter}is some{shift}text')
				\end{lstlisting}
		\end{itemize}
	\end{frame}

	\begin{frame}[fragile, allowframebreaks]{Example --- Testing the DOM and Events}
		Let's tests the DOM and some simple events
	\end{frame}



	\section{Cypress --- Integration and End-to-End Testing}
	\begin{frame}{Cypress}
		\begin{itemize}
			\item Cypress
		\end{itemize}
	\end{frame}
\end{document}
